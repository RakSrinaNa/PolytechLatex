\documentclass{polytech/polytech}
      
\usepackage{lipsum}
  
\typereport{prddi5}       

\reportyear{2015-2016}   

\title{Exemple d'utilisation de la classe de document \mbox{polytech}}
\subtitle{Il peut y avoir un sous titre mais ce n'est pas obligatoire}
\reportlogo{polytech/polytech}
           
\student[di5]{Pierre}{Premier}{pierre.premier@example.com}
\student[di4]{Paul}{Deuxième}{paul2@example.com}
\student[none]{Paul}{Deuxième}{paul2@example.com}
\student{Jacques}{Un peu plus Long}{j.unpeupluslong@example.com}

\academicsupervisor[di]{First}{Supervisor}{first.supervisor@example.com}
\academicsupervisor[dee]{Second}{Supervisor}{second.supervisor@example.com}
\academicsupervisor{Machin}{Chose}{machin.chose@example.com}

\industrialsupervisor[Fonction très importante]{Un}{Gars}{un.gars@example.com}
\industrialsupervisor{Un deuxième}{Gars}{undeuxieme.gars@example.com}

\company[polytech/polytech]{La grosse compagnie}{1 rue de l'étrange\\999999 Dans vos rèves}{example.com}

\confidential

\resume{\lipsum[1]}
\motcle{mot}
\motcle{clé}            
\motcle{deux mots}
             
\abstract{\lipsum[1-2]}
\keyword{word}
\keyword{key}
\keyword{two words}
\keyword{fourth word}


\posterblock{Objectifs}{
Lorem ipsum dolor sit amet:
\begin{itemize}
\item Vestibulum tincidunt ipsum
\item In dictum elit condimentum.
\item Suspendisse ornare orci.
\end{itemize}
Aliquam porta mi ut justo pellentesque porta.
}{polytech/polytech}{dictum elit condimentum}

\posterblock{Mise en \oe{}uvre}{
Mauris consectetur, et auctor mi \emph{fermentum}.  Etiam venenatis augue neque, ac ullamcorper sodales sit amet.
\begin{center}
\begin{tabular}{|c|c|c||c|}
\hline
1&2&3&4\\
\hline
\end{tabular}
\end{center}
Fusce aliquet sed sapien \textbf{tempor} facilisis. Nunc mi urna, cursus a ullamcorper quis, laoreet id felis.  
}{polytech/polytech}{}%Encore le logo au cas où on ne l'aurait pas vu}

\posterblock{Résultats attendus}{
\begin{itemize}
\item Quisque \textbf{dictum} diam in feugiat
\item Proin quam nec, gravida mi
\item \textbf{\emph{Suspendisse} truc bidul} vehicula magna
\item Nulla sagittis diam congue sagittis
\item Donec auctor a non nisi
\item Vestibulum quis \emph{rhoncus} risus
\end{itemize}%
}{polytech/polytech}{auctor a non nisi\\auctor a non nisi}
    
\bibliography{biblio}

\begin{document}

             
\chapter*[Titre court pour l'entête]{Introduction générale du rapport avec un titre un peu long}

\label{sec:0}    
\label{chap0}    

\begin{equation}
\label{eq:1} 
E=mc2
\end{equation}     

\begin{eqnarray}
\label{eq:a1}
E&=&mc2\\
\label{eq:a2}   
E&=&mc2\\
\label{eq:a3}
E&=&mc2 
\end{eqnarray}          
 
Equation meme chapitre : \autoref{eq:1} - \eqref{eq:1}
 
Equation meme chapitre (array) : \autoref{eq:a1} - \eqref{eq:a1}

Equation meme chapitre (array) : \autoref{eq:a2} - \eqref{eq:a2}

Equation meme chapitre (array) : \autoref{eq:a3} - \eqref{eq:a3}

Equation autre chapitre : \autoref{eq:2} - \eqref{eq:2}


Equation annexe :  \autoref{eq:3}   - \eqref{eq:3}  
      
Part : \ref{part:essai} - \autoref{part:essai} - \refchapterof{part:essai}            
       
Annexe : \ref{ann:chap1} - \autoref{ann:chap1} - \refchapterof{ann:chap1}              
 
Section d'annexe : \ref{ann:chap1:sec} - \autoref{ann:chap1:sec} - \refchapterof{ann:chap1:sec}          
 
 Chap : \ref{essai:chap1} - \autoref{essai:chap1} - \refchapterof{essai:chap1}
            
Chap section : \ref{essai:chap1:sec} - \autoref{essai:chap1:sec} - \refchapterof{essai:chap1:sec}


Chap unnumbered : \ref{essai:chap2} - \autoref{essai:chap2} - \refchapterof{essai:chap2}       

Chap unnumbered section : \ref{essai:chap2:sec} - \autoref{essai:chap2:sec} - \refchapterof{essai:chap2:sec}

This unnumbered chapter : \ref{chap0} -\autoref{chap0} - '\refchapterof{chap0}'  
  
This unnumbered chapter section: '\ref{chap0:sec}' - '\autoref{chap0:sec}' - '\refchapterof{chap0:sec}'  

This unnumbered chapter subsection: '\ref{chap0:subsec}' - '\autoref{chap0:subsec}' - '\refchapterof{chap0:sec}'

This unnumbered chapter subsubsection: '\ref{chap0:subsubsec}' - '\autoref{chap0:subsubsec}' - '\refchapterof{chap0:sec}'

This unnumbered chapter paragraph: '\ref{chap0:para}' - '\autoref{chap0:para}' - '\refchapterof{chap0:sec}'

This unnumbered chapter subparagraph: '\ref{chap0:subpara}' - '\autoref{chap0:subpara}' - '\refchapterof{chap0:sec}'
              

\lipsum[1-5]

\section{Ma section}

\label{chap0:sec}

\lipsum[1-3]

\subsection{Ma section}

\label{chap0:subsec}

\lipsum[1-3]

\subsubsection{Ma section}

\label{chap0:subsubsec}

\lipsum[1-3]

\paragraph{Ma section}

\label{chap0:para}

\lipsum[1-3]

\subparagraph{Ma section}

\label{chap0:subpara}

\lipsum[1-3]

\section{Une section n'est jamais seule, donc j'en ajoute au moins une autre et le titre peut être long si nécessaire}

\lipsum[1-3]
                       
                       
           
     
\section{Le dicton dit \og{}Jamais deux sans trois\fg{}, donc j'en ajoute au moins une autre et le titre peut être long si nécessaire mais il ne faut pas abuser car ça devient moche et peu explicite sinon}

\lipsum[1-3] 

\part{Les premiers trucs dont je vais parler}                
\label{part:premierstrucs}

\chapter{Le tout premier truc}   
\label{chap:toutpremier}

\begin{equation}
\label{eq:2} 
E=mc2
\end{equation}       
  
%\ref{sec:0}
  
%\autoref{sec:0}

\lipsum[1]
           

\part{essai}
\label{part:essai}
 
\chapter{essai:chap1}
\label{essai:chap1}
      
\lipsum[1-4]
     
\section{truc}

\label{essai:chap1:sec}

\chapter*{essai:chap2}
\label{essai:chap2}
     
\lipsum[1-4]

\section{truc}

\label{essai:chap2:sec}

  
\chapter{truc}
\lipsum[1-20]         
 
\subsection{Ma section}
\lipsum[1-5]           

\subsection{Ma section}
\lipsum[1-5]
\subsubsection{Ma section}
\lipsum[1-5]
\subsubsection{Ma section}
\lipsum[1-5]
\paragraph{Ma section}
\lipsum[1-5]
\paragraph{Ma section}
\lipsum[1-5]
\subparagraph{Ma section}
\lipsum[1-5]
\subparagraph{Ma section}
\lipsum[1-5]
  
\chapter*{Conclusion}

\appendix   

\chapter{Ma première annexe}

\label{ann:chap1}
        
\lipsum[1-4]

\begin{equation}
\label{eq:3}
E=mc2
\end{equation} 

\section{truc}
 
\label{ann:chap1:sec}

\chapter{Ma deuxième annexe}
  
\label{ann:chap2}
     
\lipsum[1-4]

\section{truc}

\label{ann:chap2:sec}

\section{Ma section}

\lipsum[1-3]

\subsection{Ma section}

\lipsum[1-3]

\subsubsection{Ma section}

\lipsum[1-3]

\paragraph{Ma section}

\lipsum[1-3]

\subparagraph{Ma section}

\lipsum[1-3]


\end{document}


